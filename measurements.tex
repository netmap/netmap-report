\subsection{The Network Performance Measurement Library}
\label{ss:measurements}
We measure more than 200 types of network data, including device-specific data such as device model, 
location data, network performance data including latency, bandwidth, and average round-trip-time, 
neighboring network infrastructure information such as neighboring network type and signal strength to 
neighboring towers/access points, and the DHCP information. All the data is stored in JSON format, so
that researchers can easily parse it.


{\bfseries Device-Specific Information.} \name{} collects device-specific information including the device ID 
(IMEI or ESN), the device type (GSM or CDMA), the software version. With the device ID, we can 
identify a device, track the device, and discard bad data that comes due to cheating. We can 
even provide personalized network usage diaries which allow users to better understand their network 
using habits.
Collecting OS version allows us to infer the effect of system and harware on the network performance. \name{} also 
logs SIM card information, such as the phone number, the SIM card operator (AT\&T, T-Mobile, etc.),
the radio type (EDGE, GPRS, HSDPA, etc.).
This permits us to create the map of network quality for each phone carrier or type, and to track how the network quality 
evolves over time.

{\bfseries Location.} The system gathers the user's location information from the GPS or network location provider. 
For the GPS, in addition to the latitude and longitude, our system also collects information 
about the observed satellites, including the almanac, ephemeris, azimuth, elevation, and SNR.
We plan to use the stallites information to detect users that cheat the game by faking their GPS locations.
Because GPS is unavailable indoors, we also use the network location provider's locations.
%gather location estimation from the network provider
%and verify its eligibility with the cell tower/AP information we collect.


{\bfseries Network Infrastructure Information.} \name{} collects information about
neighboring WiFi APs and cell towers. The cell tower information includes the mobile country code (MCC), 
mobile network code (MNC), location area code (LAC), and the signal strength; the WiFi AP information
includes the MAC address, SSID, IP address, frequency band, RSSI, link speed, etc.
With the locations where users observed the cell tower/AP, we can approximate the location of 
each cell tower/AP. Many research projects~\cite{ctrack, vtrack-sensys09} have focused on using 
energy efficient sensors to provide accurate position estimation, but most of them suffer from
low coverage problem due to lack of cell tower/AP observations. We belive we can greatly assist those projects
with the crowd-sourced data that we have. However, neighboring GSM cell tower information isn't 
universally available across all phones. Some phone models limit the cell information to only 
the connected tower or do not make it available at all.

\name{} also collects the DHCP information, including the assigned IP, network mask, IP of the DHCP server, etc.
DHCP information is useful for inferring the network topology, and for cheating detection.
 
 
{\bfseries Network Performance.} We collect sophisticated network performance measurements 
such as the network latency, bandwidth, average round-trip-time, and some TCP variables. 
For researchers, they can use the information for research on improving network reliability and performance or
 making tradeoffs when designing systems. 
For normal users, these measurements help them choose the network carrier and device which provide
 the best reception in their neighborhood. 

We use the Network Diagnostic Tool (NDT)~\cite{NDT} to collect network information. NDT measures various network 
performance metrics between the mobile device and their distributed servers. There are some straightforward 
and incredibly tedious problems in measuring the network performance. First, in order to get individual latency
measurements for both the uplink and downlink, one needs to consider the time 
synchronization problem between the device and the server. Second, to avoid variability introduced by the Internet, 
one needs to maintain servers in multiple places. 
NDT solves the time synchronization problem, and it maintains 81 servers in 27 coutries. This allows \name{}
to provide accuracy network performance measurements across the world.

\subsection{Battery/Network Awareness}
\label{ss:awareness}
There are two important aspects that designers need to take into account when implementing a mobile 
programming API. First, the API should lend itself to energy-efficient applications. 
Games using the API should avoid causing the users to incur data usage charges. 
 
\name{} addresses these two problems by incorporating the concept of battery and network aware. 
To be battery aware, \name{} monitors the battery level and status. Games using the API would get
notified when the phone is charging or when the battery level is low, and take certain actions
to save energy. For example, if the phone is charging, the game can start uploading the measurements 
it collected to the server or taking measurements more frequently. 

\name{} is also network aware. \name{} monitors the network type that the device is currently using, 
and notifies the games that using the API if the user switches to either WiFi or cellular network.
Games can react to the notification by uploading all the collected measurements to the server when
the phone is using WiFi.

