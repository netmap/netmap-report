\section{Design Overview}
\label{s:overview}

\subsection{Measurements}
We measure more than 200 types of network data, including device-specific data such as device model, 
location data, network performance data including latency, bandwidth, and average round-trip-time, 
neighboring network infrastructure information such as neighboring network type and signal strength to 
neighboring towers/access points, and the DHCP information. All the data is stored in JSON format, so
that researchers can easily parse it.


{\bfseries Device-Specific Information.} \name{} collects device-specific information including the device ID 
(IMEI or ESN), the device type (GSM or CDMA), the software version, etc. With the device ID, we can 
uniquely identify a device, track the device, and discard bad data that comes due to cheating. We can 
even provide personalized network usage diary which allows users to better understand their network 
using habits.
The software version allows us to infer the effect of the software on the network performance. \name{} also 
logs information about the SIM card, such as the phone number, and the SIM card operator (AT\&T, T-Mobile, etc.).
This permits us to create the map of network quality for each phone carrier, or how the network quality 
evolves over time.

{\bfseries Location.} The system gathers user's location information from two sources. One is the GPS:

{\bfseries Neighboring Network Infrastructure Information.} \name{} collects information about
neighboring WiFi APs and cell towers. The cell tower information includes the mobile country code (MCC), 
mobile network code (MNC), location area code (LAC), and the signal strength; the WiFi AP information
includes the MAC address, SSID, IP address, frequency band, RSSI, link speed, etc.
Along with the locations where users observed the cell tower/AP, we can approximate the location of 
each cell tower/AP. Many research projects~\cite{ctrack, vtrack-sensys09} have focused on using 
energy efficient sensors to provide accurate location, but most of them suffer from scalabity
problem due to lack of cell tower/AP observations. We belive we can greatly assist those projects
with the crowd-sourced data that we have. 

\name{} also collects the DHCP information, including the assigned IP, network mask, IP of the DHCP server, 
 
{\bfseries Network Performance.} We collect sophisticated network performance measurements between the device
and servers, such as the network latency, bandwidth, average round-trip-time, and some TCP variables. These
measurements are crucial for both network researchers and normal users. For researchers, they can use the 
information for research on improving network reliability and performance, as it will provide data on data 
transport latency and throughput for cellular and WiFi networks. For normal users, they can choose the 
network operator which provides the best reception in their neighborhood based NetMap.... 
We use the Network Diagnostic Tool (NDT)~\cite{NDT} to collect network quality information. NDT measures various network 
performance metrics between the mobile device and their distributed servers. There are some straightforward 
and incredibly tedious problems in measuring the network performance. First, one needs to consider the time 
synchronization problem between the device and the server. 
; second, to eliminate the noise in the measurement, 
one needs to maintain servers in multiple places. NDT solves the time synchronization problem, and it maintains 9 servers in the US.

\subsection{Battery/Network Aware}
There are two important aspects when it comes to designing a mobile programming API. First, the API should be energy-efficient. 
Games using the API should not drain the battery of the phone quickly; second, the API shouldn't 
 

