\subsection{Design Objectives}

Our main objectives when designing NetMap are as follows:

\textbf{Easy to integrate.} The popularity of the NetMap platform will be
inversely proportional to the cost of integrating it in a game, so we focus
our energy on reducing the surface area of the NetMap interface, and on
reducing the amount of specialized knowledge require to understand it.

\textbf{Unobtrusive.} Collecting and processing the network measurement data
should not adversely impact game performance. On mobile devices, the biggest
constraints are battery life, the bandwidth and latency of the wireless
Internet connection, and the data cap (quota) of the Internet connection.

\textbf{Resilient to failures.} Network and server failures are inevitable for
any project that runs over a non-trivial amount of time. NetMap server failures
should not negatively impact the players' experiences and, whenver possible,
should not result in measurement data loss. Game server failures should be
recoverable. To achieve this goal, we minimize the amount of shared state and
prefer idempotent API operations.

\textbf{Cheat-proof.} Any popular game has players who attempt to cheat, and
the most common way of cheating in location-based games is supplying a fake
location to the system. NetMap should be resilient to malicious players, as
well as to malicious game developers.

\textbf{Data-hungry.} To facilitate future analysis that we haven't thought
about, NetMap should collect as much network-related data as is reasonably
possible. During every measurement, we collect all the additional data that
does not impose additional burdens on the user's battery life or network
connection.
