\subsection{Algorithm}
\label{ss:algorithm}
From the large amount of data we collect, we can infer several characteristics of
wireless networks and their infrastructures. One such characteristic is the location
of cell towers that users either connect to or are in the vicinity of, since we collect
neighboring cell tower information. We tried two different algorithms for cell tower
positioning which are described below. Each algorithm uses multiple readings to
position a given cell tower. For each algorithm, we use the following abbreviations:
Location of reading = L^MLocation of cell tower = T^MDistance = L - T = d^MSignal strength = s
^M
{\bfseries Algorithm 1- Channel Model Estimation}
We begin with the common model for wireless network channels: s \propto \frac{k}{d^2}
where k is the channel condition parameter. However, we donÕt know the channel
condition parameter and it is difficult to accurately estimate. We do know that readings
in similar locations should have similar channel condition parameters so we disregard
k (since it is roughly constant across readings in same location at similar time) and state:

s \propto \frac{k}{d^2} \rightarrow d^2 = \frac{1}{s} \rightarrow d = \sqrt{\frac{1}{s}} \rightarrow L - T = \sqrt{\frac{1}{s}} \rightarrow T = L - \sqrt{\frac{1}{s}}

To obtain cell tower location, we solve for T in each reading and use the location which solves
min(sd^2 Ð 1), which is derived from minimizing d^2 = \frac{1}{s} (so sd^2 Ð 1 \approx 0)


{\bfseries Algorithm 2- Weighted Average of User Locations}
We assume that users that connect or detect a cell tower with reasonable signal strength
are relatively close to the tower. Since all user locations, after eliminating outliers, will be
a small distance from the cell tower, and each will have a different orientation with respect
to the cell tower, we weight each reading by signal strength and sum the locations to obtain
an estimate of the cell tower location.

We first, for all entries, eliminate those with signal strength below a threshold. This eliminates
the use of outliers since very low signal strength implies that the user is not very close to the
cell tower being considered. We then weight each readingÕs location (longitude and latitude separately) by:
\frac{signal strength of current reading}{sum of signal strengths from all readings above threshold}
Finally, we add the weighted longitudes and the weighted latitudes to obtain an estimate on the location
of the cell tower. The accuracy of this algorithm grows as more readings are used.


